\PassOptionsToPackage{obeyspaces}{url}

\documentclass[12pt]{beamer}
\usepackage[T1]{fontenc}
\usepackage[utf8]{inputenc}
\usepackage{graphicx}
%\usepackage[obeyspaces]{url}

\usepackage{hyperref}

\usepackage{pythontex}

\begin{document}

%\begin{frame}[fragile]
%\py{2+2}
%\begin{pygments}{py}
%a=1
%\end{pygments}
%\end{frame}

\title{Crash Course in Python}
\author{Øystein Bjørndal}
\begin{frame}[fragile]
\maketitle
\end{frame}

\begin{frame}[fragile]
\tableofcontents
\end{frame}

\section{Python 2 vs 3}
\begin{frame}[fragile]
\frametitle{Python 2 vs 3}
\begin{columns}[T]
	\begin{column}{0.48\textwidth}
Python 2 \\
\vspace{1cm}
\pygment{py}{print "hello world"}\\
\vspace{1cm}
\pygment{py}{3/2 == 1} \\
\vspace{1cm}
\begin{small}
\begin{pygments}{py}
>>> map(str.upper, ['foo', 'baz'])
['FOO', 'BAZ']
\end{pygments}
\end{small}
	\end{column}
	\begin{column}{0.48\textwidth}
Python 3 \\
\vspace{1cm}
\pygment{py}{print("hello world")}\\
\vspace{1cm}
\pygment{py}{3/2 == 1.5} \\
\vspace{1cm}
Lazy evaluation: \\
\begin{small}
\begin{pygments}{py}
>>> map(str.upper, ['foo', 'baz'])
<map at 0x2a7c10647f0>
\end{pygments}
\end{small}
	\end{column}
\end{columns}

\begin{center}
USE PYTHON 3!
\end{center}
\end{frame}

\section{Hello world}
\begin{frame}[fragile]
\frametitle{Hello world, complete how-to}
\begin{enumerate}
\item Download and install python
\url{https://www.anaconda.com/download/} \\
Anaconda contains python and a bunch of useful packages. \\
At TI, consider installing from \url{http://software.itg.ti.com/} and fixing proxy: \url{https://infolink.sc.ti.com/business_rooms/characterization_corner/f/2782/t/80190}
(first hit when searcing for "proxy" at \url{myinfolink.ti.com/}
\item Add python to Path, either during the install or manually in Windows "Environment variables for your account"
\end{enumerate}
\end{frame}

\begin{frame}[fragile]
\frametitle{Hello world, complete how-to}
\begin{enumerate}
\setcounter{enumi}{2}
\item Write a file "hello-world.py" with
\begin{pygments}{py}
print("hello world")
\end{pygments}
\item \begin{itemize}
\item  Either use Spyder (similar to the Matlab interface with a interactive session and editor)
\item use Command prompt, type "python hello-world.py" (assuming python is on your Path
and you "cd" to the correct folder).
\end{itemize}
\item "hello world" !!
\end{enumerate}
\end{frame}

\section{Some python basics}
\begin{frame}[fragile]
\frametitle{Some python basics}
Demo:\\ \begin{small}\url{https://github.com/ehmatthes/pcc/releases/download/v1.0.0/beginners_python_cheat_sheet_pcc_all.pdf}\end{small}
\end{frame}

\section{Data science libraries}
\begin{frame}[fragile]
\frametitle{Data science libraries}
Some great libraries:
\begin{itemize}
\item Matplotlib - similar plotting interface to Matlab, just better
\item Numpy - similar optimized array/matrix operations as Matlab
\item Pandas - Uses the above 2 libraries and much more, useful for data analysis, file reading/writing
\end{itemize}
\end{frame}

\subsection{Matplotlib}
\begin{frame}[fragile]
\frametitle{Data science libraries - Matplotlib}
\begin{pyblock}
import pylab as plt
# Alternative: import pylab
# Please don't: from pylab import *

x = [1, 2, 3]
y = [2, 5, 10]

plt.plot(x, y, linestyle='--', marker='*', 
	label='this is a line')
plt.legend(loc='best')
\end{pyblock}
\begin{pycode}
filename = 'fig1.pdf'
plt.savefig(filename)
plt.close(plt.gcf())
fig = r'\includegraphics[width=\textwidth]{%s}' % filename
\end{pycode}
\end{frame}

\begin{frame}[fragile]
\py{fig}
%\includegraphics[width=\textwidth]{fig1.pdf}
\end{frame}

\begin{frame}[fragile]
\frametitle{matplotlib documentation}
\url{https://matplotlib.org/api/pyplot_summary.html}
\end{frame}

\subsection{Numpy}
\begin{frame}[fragile]
\frametitle{Data science libraries - Numpy}
In the previous example, \pyv{x} and \pyv{y}, where "native" python lists, these do not work as expected from a Matlab viewpoint:
\begin{pyconsole}
x = [1, 2, 3]
y = [2, 5, 10]
x + y
x*5
# workaround
for ix in range(len(x)):
  x[ix] *= 5

x
\end{pyconsole}
\end{frame}

\begin{frame}[fragile]
\frametitle{Numpy to the rescue}
\begin{pyconsole}
import numpy as np
x = np.array([1, 2, 3])
y = np.array([2, 5, 10])
x + y
x*5
# Still works the same:
for ix in range(len(x)):
  x[ix] *= 5

x
\end{pyconsole}
\end{frame}

\begin{frame}[fragile]
\frametitle{Numpy documentation}
cheat sheet \\
\url{https://www.datacamp.com/community/blog/python-numpy-cheat-sheet} \\
\vspace{0.25cm}

Numpy for Matlab users \\
\url{https://docs.scipy.org/doc/numpy-1.14.0/user/numpy-for-matlab-users.html} \\
\vspace{0.25cm}

Documentation (or just google "numpy x")\\
\url{https://docs.scipy.org/doc/numpy-1.14.0/reference/index.html}
\end{frame}

\subsection{Pandas}
\begin{frame}[fragile]
\frametitle{Pandas documentation}
Short intro to Pandas\\
\url{https://pandas.pydata.org/pandas-docs/stable/10min.html} \\
\vspace{0.25cm}

API \\
\url{https://pandas.pydata.org/pandas-docs/stable/api.html} \\
\vspace{0.25cm}

\end{frame}


\begin{frame}[fragile]
\frametitle{Pandas}
\begin{pyblock}
import pandas as pd
s = pd.Series([2, 5, 10], index=(1000,2000,3000))
s[1000] # 2
s.plot(grid=True)
\end{pyblock}
\begin{pycode}
filename = 'fig2.pdf'
plt.savefig(filename)
fig2 = r'\includegraphics[width=\textwidth]{%s}' % filename
\end{pycode}
\end{frame}

\begin{frame}[fragile]
\py{fig2}
\end{frame}

\begin{frame}[fragile]
\frametitle{Pandas indexing series}
\begin{pyconsole}
import pandas as pd
s = pd.Series([2, 5, 10], index=(1000,2000,3000))
s[1000]
s.iloc[0]
s[3000]
s.iloc[-1]
\end{pyconsole}
\end{frame}

\begin{frame}[fragile]
\frametitle{Pandas indexing series, using strings}
\begin{pyconsole}
import pandas as pd
s = pd.Series([2, 5, 10], index=('Experiment 1',
					'Debug',
					'Debug-Debug'))
s['Experiment 1']
s.iloc[0]
s['Debug-Debug']
s.iloc[-1]
\end{pyconsole}
\end{frame}

\begin{frame}[fragile]
\frametitle{Calling from LabView}
\url{https://oslosvn.norway.design.ti.com/svn/lab/Kharon LabView 2016/trunk/Register files/Auto update register files/call_python.vi}

\vspace{1cm}
Example usage
\begin{itemize}
\item \url{https://oslosvn.norway.design.ti.com/svn/lab/Kharon LabView 2016/trunk/User Lib/Generic instrument methods/General/Log_Instr_excel.vi}
\item \url{https://oslosvn.norway.design.ti.com/svn/lab/Kharon LabView 2016/trunk/User Lib/Utilities/Reporting/Report tool/Panels/MergeToSpotfire/concatenate_csv_MAIN.vi}
\end{itemize}
\end{frame}

\section{Style guide}
\begin{frame}[fragile]
\frametitle{PEP 8}
Write better code
\url{https://www.python.org/dev/peps/pep-0008/?}\\
Not intended for beginner python users, but shows some interesting syntax and use cases and
describes best-practices.

\end{frame}

\begin{frame}[fragile]
\begin{footnotesize}
\begin{pyconsole}
import this
\end{pyconsole}
\end{footnotesize}
\end{frame}
\end{document}