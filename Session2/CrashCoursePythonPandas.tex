\PassOptionsToPackage{obeyspaces}{url}

\documentclass[12pt]{beamer}
\usepackage[T1]{fontenc}
\usepackage[utf8]{inputenc}
\usepackage{graphicx}
%\usepackage[obeyspaces]{url}

\usepackage{hyperref}

\usepackage{pythontex}

\begin{document}

%\begin{frame}[fragile]
%\py{2+2}
%\begin{pygments}{py}
%a=1
%\end{pygments}
%\end{frame}

\title{Crash Course in Python}
\author{Øystein Bjørndal}
\begin{frame}[fragile]
\maketitle
\end{frame}

\begin{frame}[fragile]
\tableofcontents
\end{frame}

\section{Hello world}
\begin{frame}[fragile]
\frametitle{Hello world, complete how-to}
\begin{enumerate}
\item Download and install python
\url{https://www.anaconda.com/download/} \\
Anaconda contains python and a bunch of useful packages. \\
At TI, consider installing from \url{http://software.itg.ti.com/} and fixing proxy: \url{https://infolink.sc.ti.com/business_rooms/characterization_corner/f/2782/t/80190}
(first hit when searcing for "proxy" at \url{myinfolink.ti.com/}
\item Add python to Path, either during the install or manually in Windows "Environment variables for your account"
\end{enumerate}
\end{frame}

\begin{frame}[fragile]
\frametitle{Hello world, complete how-to}
\begin{enumerate}
\setcounter{enumi}{2}
\item Write a file "hello-world.py" with
\begin{pygments}{py}
print("hello world")
\end{pygments}
\item \begin{itemize}
\item  Either use Spyder (similar to the Matlab interface with a interactive session and editor)
\item use Command prompt, type "python hello-world.py" (assuming python is on your Path
and you "cd" to the correct folder).
\end{itemize}
\item "hello world" !!
\end{enumerate}
\end{frame}

\section{Some python basics}
\begin{frame}[fragile]
\frametitle{Some python basics}
Demo:\\ \begin{small}\url{https://github.com/ehmatthes/pcc/releases/download/v1.0.0/beginners_python_cheat_sheet_pcc_all.pdf}\end{small}
\end{frame}

\section{Data science libraries}
\begin{frame}[fragile]
\frametitle{Data science libraries}
Some great libraries:
\begin{itemize}
\item Matplotlib - similar plotting interface to Matlab, just better
\item Numpy - similar optimized array/matrix operations as Matlab
\item Pandas - Uses the above 2 libraries and much more, useful for data analysis, file reading/writing
\end{itemize}
\end{frame}


\subsection{Pandas}
\begin{frame}[fragile]
\frametitle{Pandas documentation}
Short intro to Pandas\\
\url{https://pandas.pydata.org/pandas-docs/stable/10min.html} \\
\vspace{0.25cm}

API \\
\url{https://pandas.pydata.org/pandas-docs/stable/api.html} \\
\vspace{0.25cm}

\end{frame}


\begin{frame}[fragile]
\frametitle{Pandas}
\begin{pyblock}
import pandas as pd
df = pd.DataFrame([[2, 5, 10], [3, 6, 11]], 
			columns=('col1', 'col2', 'col3'))
df['col1'] #  [2, 3]
df.plot('col1', 'col2', grid=True)
\end{pyblock}
\begin{pycode}
import pylab as plt
filename = 'fig2.pdf'
plt.savefig(filename)
fig2 = r'\includegraphics[width=\textwidth]{%s}' % filename
\end{pycode}
\end{frame}

\begin{frame}[fragile]
\py{fig2}
\end{frame}

\subsubsection{Plotting with Pylab}
\begin{frame}[fragile]
\frametitle{Plotting with Pylab}
\begin{pyblock}
import pylab as plt
import pandas as pd

df = pd.DataFrame([[2, 5, 10], [3, 6, 11]], 
			columns=('col1', 'col2', 'col3'))
df['col1'] #  [2, 3]

plt.figure(2)
plt.plot(df['col1'].values, df['col2'].values,
	marker='.')
plt.grid(True)
\end{pyblock}
\begin{pycode}

filename = 'fig22.pdf'
plt.savefig(filename)
fig22 = r'\includegraphics[width=\textwidth]{%s}' % filename
\end{pycode}
\end{frame}

\begin{frame}[fragile]
\py{fig22}
\end{frame}

\subsubsection{Pandas indexing}
\begin{frame}[fragile]
\frametitle{Pandas indexing}
\begin{pyconsole}
import pandas as pd

df = pd.DataFrame([[2, 5, 10], [3, 6, 11]], 
			columns=('col1', 'col2', 'col3'))
df['col1']
df['col1'].iloc[0]
df['col1'].iloc[1]
df['col1'].iloc[-1] # last element
\end{pyconsole}
\end{frame}

\subsubsection{Pandas indexing slize}
\begin{frame}[fragile]
\frametitle{Pandas indexing slize}
\begin{pyconsole}
import pandas as pd

df = pd.DataFrame([[2, 5, 10], [3, 6, 11]], 
			columns=('col1', 'col2', 'col3'))
df
# start at index 0, end at index 2 (not inclusive)
df['col1'].iloc[0:2] 
\end{pyconsole}
\end{frame}

\subsubsection{Pandas return submatch}
\begin{frame}[fragile]
\frametitle{Pandas return submatch}
\begin{pyconsole}
df = pd.DataFrame([[2, 5, 10], [3, 6, 11], [4, 5, 12]], 
			columns=('col1', 'col2', 'col3'))
df
df['col2'] == 5
df[df['col2'] == 5] 
\end{pyconsole}
\end{frame}

\subsubsection{Pandas reading and writing to excel}
\begin{frame}[fragile]
\frametitle{Pandas reading and writing to excel}
\begin{pyconsole}
root = r'R:\CC2652R\PG2.1_Q1\Debug\OB_OPT2\TX_BLE_Quality\CC2652REM-7ID-Q1\Test Bench 14'
root_tmp = r'c:\tmp'
filename = 'TX_BLE_quality_frequencySweep_TC029_TxPower_003F_DCDC_ON_spotfire.xlsx'
full_path_in = os.path.join(root, filename)
full_path_out = os.path.join(root_tmp, filename)

 # read
df = pd.read_excel(full_path_in)
# write
df.to_excel(full_path_out, index=False)
# see 
# svn\lab\Python\py_excel_utilities\pandas_util.py
# for more formats
\end{pyconsole}
\end{frame}

\begin{frame}[fragile]
\subsubsection{Getting filenames}
\frametitle{Getting filenames}
\begin{pyconsole}
# single filename
full_path = r'c:\tmp\myfile.xlsx'
# OR
root = r'c:\tmp'
filename = 'myfile.xlsx'
full_path = os.path.join(root, filename)
\end{pyconsole}
\end{frame}

\begin{frame}[fragile]
\frametitle{Getting files continued}
\begin{pyconsole}
# multiple filenames, wildcard
from glob import glob
for full_path in glob(r'c:\tmp\*.xlsx'):
  print(full_path)
  break;# break loop
  
# multiple filenames, all subfolders
for root, dirs, files in os.walk(r'c:\tmp'):
  for filename in files:
    if filename.endswith('.xlsx'):
      full_path = os.path.join(root, filename)

\end{pyconsole}
\end{frame}

\subsubsection{DataFrame Grouping}
\begin{frame}[fragile]
\frametitle{DataFrame Grouping}
\begin{pyconsole}
df['EM name'].unique()
for temperature, group in df.groupby('Temperature'):
  for em_name, group in group.groupby('EM name'):
    print(temperature, em_name)

group['EM name'].unique()
\end{pyconsole}
\end{frame}

\section{Demo}
\begin{frame}[fragile]
\frametitle{Demo}
\url{https://dosfiler01.itg.ti.com/svn/lab/Python/PythonCrashCourse/cleanup_tof_results.py}
\end{frame}

\section{Style guide}
\begin{frame}[fragile]
\frametitle{PEP 8}
Write better code
\url{https://www.python.org/dev/peps/pep-0008/?}\\
Not intended for beginner python users, but shows some interesting syntax and use cases and
describes best-practices.

\end{frame}

\begin{frame}[fragile]
\begin{footnotesize}
\begin{pyconsole}
import this
\end{pyconsole}
\end{footnotesize}
\end{frame}
\end{document}